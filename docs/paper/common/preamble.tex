% Recommended, but optional, packages for figures and better typesetting:
\usepackage{microtype}
\usepackage{graphicx}
\usepackage{booktabs} % for professional tables

% hyperref makes hyperlinks in the resulting PDF.
% If your build breaks (sometimes temporarily if a hyperlink spans a page)
% please comment out the following usepackage line and replace
% \usepackage{icml2021} with \usepackage[nohyperref]{icml2021} above.
\usepackage{hyperref}

% Attempt to make hyperref and algorithmic work together better:
\newcommand{\theHalgorithm}{\arabic{algorithm}}

% % % Use the following line for the initial blind version submitted for review:
% \usepackage{icml2021}

% If accepted, instead use the following line for the camera-ready submission:
\usepackage[accepted]{icml2021}

% The \icmltitle you define below is probably too long as a header.
% Therefore, a short form for the running title is supplied here:
\icmltitlerunning{Robust Asymmetric Learning in POMDPs}

% AW - 
\usepackage{caption}
\usepackage{subcaption}
\usepackage{amsthm}
\usepackage{multirow}
\usepackage{amsmath}
\usepackage{amsfonts}
\usepackage{pgfplots}
\usepackage{lipsum}
\usepackage{natbib}
\usepackage{listings}
\usepackage{amssymb}
\usepackage{algorithm}
\usepackage{algorithmic,eqparbox,array}
\usepackage{balance}
\usepackage{chngpage}
\usepackage{setspace}
\usepackage{etoolbox}
\usepackage{tikz}
\usetikzlibrary{shapes.geometric}
\usetikzlibrary{calc}
\usetikzlibrary{shapes,arrows}

\pgfplotsset{compat=1.16}

% Define new theorem macros.
\newtheorem{assumption}{Assumption}
\newtheorem{asscorollary}{Corollary}[assumption]
\newtheorem{theorem}{Theorem}%[section]
\newtheorem{lemma}{Lemma}
\newtheorem{proposition}{Proposition}
\newtheorem{corollary}{Corollary}
\newtheorem{definition}{Definition}
\newtheorem{asump}{Assumption}

% Stuff for todo notes.
\usepackage[colorinlistoftodos,prependcaption, textsize=tiny,textwidth=2cm]{todonotes}
\newcommand{\unsure}[2][]{\todo[linecolor=red,backgroundcolor=red!25,bordercolor=red,#1]{#2}}
\newcommand{\tocite}[2][]{\todo[linecolor=blue,backgroundcolor=blue!25,bordercolor=blue,#1]{#2}}
\newcommand{\improvement}[2][]{\todo[linecolor=yellow,backgroundcolor=yellow!25,bordercolor=yellow,#1]{#2}}
\newcommand{\implement}[2][]{\todo[linecolor=green,backgroundcolor=green!25,bordercolor=green,#1]{#2}}
\newcommand{\aw}[1]{\textcolor{blue}{[AW: #1]}}
\newcommand{\fw}[1]{\textcolor{green}{[FW: #1]}}
\newcommand{\wl}[1]{\textcolor{cyan}{[WL: #1]}}
\renewcommand{\check}[1]{\textcolor{red}{[CHECK: #1]}}
\newcommand{\cw}[1]{\textcolor{green}{[CW: #1]}}
\newcommand{\as}[1]{\textcolor{green}{[AS: #1]}}

% Some math macros.
\newcommand{\tends}{\rightarrow}
\DeclareMathOperator{\argmax}{arg\,max}
\DeclareMathOperator{\argmin}{arg\,min}
\newcommand{\where}{\quad \text{where }\ \ }

% Counters for hacking theorem numbers.
\newcounter{sthe}
\newcounter{sequ}
    
\usepackage{enumitem}
\setlist[enumerate]{itemsep=0mm}

\usepackage{enumitem}
\setlist{leftmargin=5.5mm}